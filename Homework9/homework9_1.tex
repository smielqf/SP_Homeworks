\documentclass{article}
\usepackage{ctex}
\usepackage{graphicx}
\usepackage{ctex}
\usepackage{amsmath}
\usepackage{amsfonts}
\usepackage{amssymb}
\usepackage{enumerate}
\usepackage{color}
\usepackage{setspace}
\usepackage{pythonhighlight}
\usepackage{bm}

\usepackage
[a4paper,
text={146.4true mm,239.2 true mm},
top= 26.2true mm,
left=31.8 true mm,
head=6true mm,
headsep=6.5true mm,
foot=16.5true mm]
{geometry} % 设置文本的边距
\input{../setup/format}

\begin{document}
    \title{Homework 1 of Dynamic Programming and Optimal Control}
    \author{姓名:林奇峰\qquad 学号:19110977}
    \maketitle

    \section{Exercise 1.1}
    Consider the system
    \begin{equation*}
        x_{k+1}=x_k+u_k+w_k,\qquad k=0,1,2,3
    \end{equation*}
    
    with initial state $x_0=5$, and the cost function
    \begin{equation*}
        \sum_{k=0}^3(x_k^2+u_k^2)
    \end{equation*}
    
    Apply the DP algorithm for the following three cases:
    \begin{enumerate}[(a)]
        \item The control constraint set $U_k(x_k)$ is $\{u|0\leq x_k+u\leq5, u:\text{ integer}\}$ for all $x_k$ and $k$, and the disturbance $w_k$ is equal to zero for all $k$. 
    \end{enumerate}
    \begin{enumerate}[(c)]
        \item The control constraint is as in part (a) and the disturbance $w_k$ takes the values -1 and 1 with equal probability 1/2 for all $x_k$ and $u_k$, except if $x_k+u_k$ is equal to 0 or 5,  in which case $w_k=0$ with probability 1.
    \end{enumerate}
    \textbf{Solutions:}
    \begin{enumerate}[(a)]
        \item 由题可知,$N=4$且$g_N(x_N)=0$。由于$w_k=0 \text{ for all }k$, $x_{k+1}=x_k+u_k+w_k=x_k+u_k$,且
            \begin{equation}
                \begin{split}
                    &\min_{u_k\in U_k(x_k)}\mathop{E}_{w_k}\big\{g_k(x_k,u_k,w_k)+J_{k+1}(f_k(x_k, u_k, w_k))\big\}\\
                    =&\min_{u_k\in U_k(x_k)}g_k(x_k,u_k)+J_{k+1}(f_k(x_k, u_k))\\
                    =&\min_{u_k\in U_k(x_k)}g_k(x_k,u_k)+J_{k+1}(x_k+u_k)\\
                \end{split}
            \end{equation}
        
            另外,代价函数$\sum_{k=0}^3(x_k^2+u_k^2)=\sum_{k=0}^3g_k(x_k,u_k)$可以明显地通过最小化每个子问题$g_k(x_k,u_k)$来最小化本身,符合动态规划的使用场景。因此我们使用动态规划的方法来解决本问题。    

            从约束$\{u|0\leq x_k+u_k\leq 5, u:\text{ integer}\}$中我们可以得到$u$的取值集合为$U_k(x_k)=\{-x_k, 1-x_k, 2-x_k, 3-x_k, 4-x_k, 5-x_k\}$。
        
            则,当$N=4$时
            \begin{equation}
                J_4(x_4)=g_4(x_4)=0
            \end{equation}

        
            当$k=3$时,
            \begin{equation}
                \begin{split}
                    J_3(x_3)&=\min_{u_3\in U_3(x_3)}g_3(x_3,u_3)+J_4(x_3+u_3)\\
                    &=\min_{u_3\in U_3(x_3)}x_3^2+u_3^2
                \end{split}
            \end{equation}
            令$h_3(u_3)=x_3^2+u_3^2$,并对其求导可得
            \begin{equation}
                \begin{split}
                    h_3(u_3)'=2u_3
                \end{split}
            \end{equation}
            由于存在约束$\{u|0\leq x_3+u_3\leq 5, u:\text{ integer}\}$,我们要分情况讨论。
            \begin{enumerate}
                \item 当$x_3<0$时,$u_3>0$且$h_3(u^3)'=2u_3>0$。因此$u_3$的取值越小越好,即$u_3^*=-x_3$。
                \item 当$x_3>5$时,$u_3<0$且$h_3(u^3)'=2u_3<0$。因此$u_3$的值越大越好,即$u_3^*=5-x_3$。
                \item 当$0\leq x_3\leq 5$时,最小值为$h_3^*(u_3)=0$且$u_3^*=0$。
            \end{enumerate}
            将上述结果代回去可以得到
            \begin{equation}
                J_3(x_3)=
                \begin{cases}
                    x_3^2+u_3^2 & ,x_3\leq 0\\
                    x_3^2 & ,0\leq x_3\leq 5\\
                    2x_3^2-10x_3+25 & ,x_3>5
                \end{cases}
            \end{equation}

            当$k=2$时,$0\leq x_3=x_2+u_2\leq 5$,因此
            \begin{equation}
                \begin{split}
                    J_2(x_2) &=\min_{u_2\in U(x_2)}g_2(x_2,u_2)+J_3(x_2+u_2)\\
                    &=\min_{u_2\in U(x_2)}x_2^2+u_2^2+(x_2+u_2)^2
                \end{split}
            \end{equation}
            令$h_2(u_2)=x_2^2+u_2^2+(x_2+u_2)^2$,并对其求导可得
            \begin{equation}
                h_2(u_2)'=4u_2+2x_2=4(x_2+u_2)-2x_2
            \end{equation}
            由于存在约束$\{u|0\leq x_2+u_2\leq 5, u:\text{ integer}\}$,我们要分情况讨论。
            \begin{enumerate}
                \item 当$x_2>10$时,$h_2(u_2)'<0$,因此$u_2$的值越大越好,即$u_2^*=5-x_2$
                \item 当$x_2<0$时,$h_2(u_2)'>0$,因此$u_2$的取值越小越好,即$u_2^*=-x_2$
                \item 当$0\leq x_2\leq 10$时,最小值出现在$h'=0$的时候
                    \begin{enumerate}
                        \item 当$x_2$为偶数的时候,$u_2^*=-\frac{x_2}{2}$
                        \item 当$x_3$为奇数的时候,$u_2^*=-\frac{x_2+1}{2}$或者$u_2^*=-\frac{x_2-1}{2}$,因为$h$是偶函数。
                    \end{enumerate}
            \end{enumerate}
            将上述结果代回去可以得到
            \begin{equation}
                J_2(x_2)=
                \begin{cases}
                    2x_2^2 & ,x_2<0\\
                    \frac{3}{2}x_2^2+\frac{1}{2}\cdot\bm{1}_{x_2\text{ is odd} } &, 0\leq x_2\leq 10\\
                    2x_2^2-10x_2+50 &,x>10
                \end{cases}
            \end{equation}

            当$k=1$时,$0\leq x_2=x_1+u_1\leq 5$,因此
            \begin{equation}
                \begin{split}
                    J_1(x_1) &=\min_{u_1\in U(x_1)}g_1(x_1,u_1)+J_2(x_1+u_1)\\
                    &=\min_{u_1\in U(x_1)}x_1^2+u_1^2+\frac{3}{2}(x_1+u_2)^2+\frac{1}{2}\cdot\bm{1}_{x_1+u_1\text{ is odd}}
                \end{split}
            \end{equation}
            令$h_1(u_1)=x_1^2+u_1^2+\frac{3}{2}(x_1+u_2)^2+\frac{1}{2}\cdot\bm{1}_{x_1+u_1\text{ is odd}}$,并对其求导可得
            \begin{equation}
                h_1(u_1)'=5u_1+3x_1=5(x_1+u_1)-2x_1
            \end{equation}
            由于存在约束$\{u|0\leq x_1+u_1\leq 5, u:\text{ integer}\}$,且$h_1'$有点难以直接求解。因此,直接列出可能的取值以求解。
            \begin{equation}
                \begin{aligned}
                    h_1(-x_1) &= 2x^2_1\\
                    h_1(1-x_1) &= 2x^2_1-2x_1+3\\
                    h_1(2-x_1) &= 2x^2_1-4x_1+10\\
                    h_1(3-x_1) &= 2x^2_1-6x_1+23\\
                    h_1(4-x_1) &= 2x^2_1-8x_1+40\\
                    h_1(5-x_1) &= 2x^2_1-10x_1+63
                \end{aligned}
            \end{equation}
            对上述式子联立求区间可得最终结果
            \begin{equation}
                J_1(x_1)=
                \begin{cases}
                    2x^2_1 &, u_1^*=-x_1\text{ and } x_1\leq\frac{3}{2}\\
                    2x^2_1-2x_1+3 &, u_1^*=1-x_1\text{ and } \frac{3}{2}<x_1\leq\frac{7}{2}\\
                    2x^2_1-4x_1+10 &, u_1^*=2-x_1\text{ and } \frac{7}{2}<x_1\leq\frac{13}{2}\\
                    2x^2_1-6x_1+23 &, u_1^*=3-x_1\text{ and } \frac{13}{2}<x_1\leq\frac{17}{2}\\
                    2x^2_1-8x_1+40 &, u_1^*=4-x_1\text{ and } \frac{17}{2}<x_1\leq\frac{23}{2} \\
                    2x^2_1-10x_1+63 &, u_1^*=5-x_1\text{ and } x_1>\frac{23}{2}
                \end{cases}
            \end{equation}

            当$k=0$时,$0\leq x_1=x_0+u_0\leq 5$且$x_0=5,-5\leq u_0\leq 0$,因此
            \begin{equation}
                \begin{split}
                    J_0(x_0) &=\min_{u_0\in U_{(x_0)}}g_0(x_0,u_0)+J_1(x_0+u_0)\\
                    &=25+u_0^2+J_1(5+u_0)
                \end{split}
            \end{equation}
            同样,令$h_0(u_0)=25+u_0^2+J_1(5+u_0)$,并对$u_0$进行取值可得
            \begin{equation}
                \begin{aligned}
                    h_0(-5)&=50\\
                    h_0(-4)&=43\\
                    h_0(-3)&=41\\
                    h_0(-2)&=44\\
                    h_0(-1)&=52\\
                    h_0(0)&=65
                \end{aligned}
            \end{equation}
            则$u^*_0=-3$且$J_0(x_0)=41$。
    \end{enumerate}
   

    \section{Exercise 2.4 (Dijkstra's Algorithm for Shortest Paths)}
    Consider the best-first version of the label correcting algorithm of Section 2.3.1. Here at each iteration we remove from OPEN a node that has minimum label over all nodes in OPEN.
    \begin{enumerate}[(a)]
        \item Show that each node $j$ will enter OPEN at most once, and show that at the time it exits OPEN, its label $d_j$ is equal to the shortest distance from $s$ to $j$. \textit{Hint:} Use the nonnegative arc length assumption to argue that in the label correcting algorithm, in order for the node $i$ that exist OPEN to reenter, there must exist another node $k$ in OPEN with $d_k+a_{ki}<d_i$.
        \item Show that the number of arithmetic operations required for termination is bounded by $cN^2$ where $N$ is the number of nodes and $c$ is some constant.
    \end{enumerate}
    \textbf{Solutions:}
\end{document}