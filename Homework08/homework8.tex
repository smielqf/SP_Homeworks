\documentclass{article}
\usepackage{ctex}
\usepackage{graphicx}
\usepackage{ctex}
\usepackage{amsmath}
\usepackage{amsfonts}
\usepackage{amssymb}
\usepackage{enumerate}
\usepackage{color}
\usepackage{setspace}
\usepackage{pythonhighlight}
\usepackage{bm}

\usepackage
[a4paper,
text={146.4true mm,239.2 true mm},
top= 26.2true mm,
left=31.8 true mm,
head=6true mm,
headsep=6.5true mm,
foot=16.5true mm]
{geometry} % 设置文本的边距
\input{../setup/format}

\begin{document}
    \title{Homework 8 of Stochastic Processes}
    \author{姓名:林奇峰\qquad 学号:19110977}
    \maketitle

    \section{Exercise 5.6}
    A town starts a mosquito control program and rv $Z_n$ is the number of mosquitos at the end of the $n$th year $(n=0,1,2,\dots)$. Let $X_n$ be the growth rate of the mosquito population in year $n$; i.e., $Z_n=X_nZ_{n-1}$; $n\geq 1$. Assume that $\{X_n;n\geq 1\}$ is a sequence of IID rv s with the PMF $\text{Pr}\{X=2\}=1/2$; $\text{Pr}\{X=1/2\}=1/4$; $\text{Pr}\{1/4\}=1/4$. Suppose that $Z_0$, the initial number of mosquitos, is some known constant and assume for simplicity and consistency that $Z_n$ can take on non-integer values.
    \begin{enumerate}[(a)]
        \item Find E$[Z_n]$ as a function of $n$ and find $\lim_{n\rightarrow\infty}$E$[Z_n]$.
        \item Let $W_n=\log_2X_n$. Find E$[W_n]$ and E$[\log_2(Z_n/Z_0)]$ as a function of $n$.
        \item There is a constant $\alpha$ such that $\lim_{n\rightarrow\infty}(1/n)[\log_2(Z_n/Z_0)]=\alpha$ WP1. Find $\alpha$ and explain how this follows from the SLLN.
        \item Using (c), show that $\lim_{n\rightarrow\infty}Z_n=\beta$ WP1 for some $\beta$ and evaluate $\beta$.
        \item Explain carefully how the result in (a) and the result in (d) are compatible. What you should learn from this problem is that the expected value of the log of a product of IID rv s might be significant than the expected value of the product itself.
    \end{enumerate}

    \textbf{Solutions:}
    \begin{enumerate}[(a)]
        \item 
            \begin{equation*}
                \begin{split}
                    \text{E}[Z_n] &= \text{E}[Z_0\prod_{i=1}^nX_i]\\
                    &=Z_0\prod_{i=1}^n\text{E}[X_i]\qquad\text{from the fact that $\{X_n;n\geq 1\}$ is a sequence of IID rv s}\\
                    &=Z_0\cdot\bigg(\frac{19}{16}\bigg)^n
                \end{split}
            \end{equation*}
            since each rv $X_i$ is IID and E$[X_i]=2\cdot\frac{1}{2}+\frac{1}{2}\cdot\frac{1}{2}+\frac{1}{4}\cdot\frac{1}{4}=\frac{19}{16}$.

            \begin{equation*}
                \begin{split}
                    \lim_{n\rightarrow\infty}
                    \text{E}[Z_n]=\lim_{n\rightarrow\infty}Z_0\cdot\bigg(\frac{19}{16}\bigg)^n=\infty
                \end{split}
            \end{equation*}
        \item 
            \begin{equation*}
                \begin{split}
                    \text{E}[W_n] &= \frac{1}{2}\cdot\log_22+\frac{1}{4}\cdot\log_2\frac{1}{2}+\frac{1}{4}\cdot\log_2\frac{1}{4}\\
                    &=-\frac{1}{4}
                \end{split}
            \end{equation*}

            \begin{equation*}
                \begin{aligned}
                    Z_n/Z_0&=X_n\cdot X_{n-1}\cdot X_1\cdot Z_0/Z_0&=\prod_{i=1}^nX_i\\
                    &\Downarrow\\
                    \log_2(Z_n/Z_0)&=\sum_{i=1}^n\log_2X_i\\
                    &=\sum_{i=1}^nW_i\\
                    &\Downarrow\\
                    \text{E}[\log_2(Z_n/Z_0)] &= \sum_{i=1}^n\text{E}[W_i]\\
                    &=-\frac{n}{4}
                \end{aligned}
            \end{equation*}
            From the fact that $\{X_n;n\geq 1\}$ is a sequence of IID rv s and thus ${W_n;n\geq 1}$ is also a sequence of IID rv s.
        \item Since
            \begin{equation*}
                \begin{split}
                    \lim_{n\rightarrow\infty}(1/n)[\log_2(Z_n/Z_0)] &= \lim_{n\rightarrow\infty}\frac{1}{n}\cdot-\frac{n}{4}\\
                    &=\lim_{n\rightarrow\infty}-\frac{1}{4}\\
                    &=-\frac{1}{4}
                \end{split}
            \end{equation*}
            We can obtain that $\alpha=-\frac{1}{4}$.

            According to Theorem 5.2.3 (SLLN), \textit{For each integer $n\geq 1$, let $S_n=X_1+\cdots+X_n$, where $X_1,X_2,\dots$ are IID rv s satisfying \textnormal{E}$[|X|]<\infty$. Then}
            \begin{equation*}
                \text{Pr}\bigg\{\omega:\lim_{n\rightarrow\infty}\frac{S_n(\omega)}{n}=\overline{X}\bigg\}=1
            \end{equation*}

            We can see that, under this case, $S_n=\sum_{i=1}^nW_i$ where $W_i$ is IID rv and it satisfies E$[|W_i|]=-\frac{1}{4}<\infty$. The result shows that 
            \begin{equation*}
                \text{Pr}\bigg\{\omega:\lim_{n\rightarrow\infty}\frac{S_n(\omega)}{n}=\overline{X}=-\frac{1}{4}\bigg\}=1
            \end{equation*}
            holds.
        \item TO-DO
        \item TO-DO 
    \end{enumerate}
\end{document}