\documentclass{article}
\usepackage{ctex}
\usepackage{graphicx}
\usepackage{ctex}
\usepackage{amsmath}
\usepackage{amsfonts}
\usepackage{amssymb}
\usepackage{enumerate}
\usepackage{color}
\usepackage{setspace}
\usepackage{pythonhighlight}
\usepackage{bm}

\usepackage
[a4paper,
text={146.4true mm,239.2 true mm},
top= 26.2true mm,
left=31.8 true mm,
head=6true mm,
headsep=6.5true mm,
foot=16.5true mm]
{geometry} % 设置文本的边距
\input{../setup/format}

\begin{document}
    \title{Homework 2 of Dynamic Programming and Optimal Control}
    \author{姓名:林奇峰\qquad 学号:19110977}
    \maketitle

    \section{Exercise 9.6}
    Define $\gamma(r)$ as $\ln[\text{g}(r)]$ where $\text{g}(r)=\text{E}[\exp(rX)]$. Assume that $X$ is discrete with possible outcomes $\{a_i;i\geq1\}$, let $p_i$ denote $\text{Pr}\{X=a_i\}$, and assume that $\text{g}(r)$ exists in some open interval $(r_\_,r_+)$ containing $r=0$. For any given $r,r_\_<r<r_+$, define a rv $X_r$ with the same set of possible outcomes $\{a_i;i\geq1\}$ as $X$, but with a PMF $q_i=\text{Pr}\{X_r=a_i\}=p_i\exp[a_ir-\gamma(r)]$. Note that $X_r$ is not a function of $X$ and is not even to be viewed as in the same probability space as $X$; it is of interest simply because of the behavior of its defined probability mass function. It is called a tilted rv relative to $X$, and this exercise, along with Exercise 9.11, will justify our interest in it.
    \begin{enumerate}[(a)]
        \item Verify that $\sum_iq_i=1$.
        \item Verify that $\text{E}[X_r]=\sum_ia_iq_i$ is equal to $\gamma'(r)$.
        \item Verify that $\text{VAR}[X_r]=\sum_ia_i^2q_i-(\text{E}[X_r])^2$ is equal to $\gamma''(r)$.
        \item Argue that $\gamma''(r)\geq0$ for all $r$ such that $\text{g}(r)$ exists, and that $\gamma''(r)>0$ if $\gamma''(0)>0$.
    \end{enumerate}
    \textbf{Solutions:}
    \begin{enumerate}[(a)]
        \item \begin{equation*}
            \sum_iq_i=\sum_ip_i\exp[a_ir-\gamma(r)]=\sum_i\frac{p_i\exp[a_ir]}{\text{g}(r)}=\frac{\sum_ip_i\exp[a_ir]}{\text{g}(r)}=\frac{\text{g}(r)}{\text{g}(r)}=1
        \end{equation*}
        \item \begin{equation*}
            \begin{split}
                \text{E}[X_r]&=\sum_ia_iq_i\\
                &=\sum_ia_ip_i\exp[a_ir-\gamma(r)]\\
                &=\frac{\sum_ip_ia_i\exp[a_ir]}{\text{g}(r)}\\
                &=\frac{1}{{\text{g}(r)}}\frac{d\sum_ip_i\exp[a_ir]}{dr}\\
                &=\frac{\text{g}'(r)}{\text{g}(r)}=\gamma'(r)
            \end{split}
        \end{equation*}
        \item Since
        \begin{equation*}
            \begin{split}
                \sum_ia_i^2q_i &= \sum_ia_i^2p_i\exp[a_ir-\gamma(r)]\\
                &=\frac{1}{\text{g}(r)}\sum_ip_ia_i^2\exp[a_ir]\\
                &=\frac{1}{\text{g}(r)}\frac{d\sum_ip_ia_i\exp[a_ir]}{dr}\\
                &=\frac{1}{\text{g}(r)}\frac{d\text{g}'(r)}{dr}\\
                &=\frac{\text{g}''(r)}{\text{g}(r)}
            \end{split}
        \end{equation*}

        we have
        \begin{equation*}
            \begin{split}
                \text{VAR}[X_r] &=\sum_ia_i^2q_i-(\text{E}[X_r])^2\\
                &=\frac{\text{g}''(r)}{\text{g}(r)}-\frac{[\text{g}'(r)]^2}{[\text{g}(r)]^2}\\
                &=\frac{\text{g}''(r)-[\text{g}(r)]^2}{[\text{g}(r)]^2}\\
                &=\gamma''(r)
            \end{split}
        \end{equation*}
        \item Since $\gamma''(r)=\text{VAR}[X_r]$, it is nonnegative and $\gamma''(r)\geq0$. If $\gamma''(0)>0$, then $\text{VAR}[X]>0$, which means that $X$ is non-atomic. Thus $X_r$ is non-atomic and $\gamma''(r)>0$.
    \end{enumerate}
\end{document}