\documentclass{article}
\usepackage{ctex}
\usepackage{graphicx}
\usepackage{ctex}
\usepackage{amsmath}
\usepackage{amsfonts}
\usepackage{amssymb}
\usepackage{enumerate}
\usepackage{color}
\usepackage{setspace}
\usepackage{pythonhighlight}
\usepackage{bm}

\usepackage
[a4paper,
text={146.4true mm,239.2 true mm},
top= 26.2true mm,
left=31.8 true mm,
head=6true mm,
headsep=6.5true mm,
foot=16.5true mm]
{geometry} % 设置文本的边距
\input{../setup/format}

\begin{document}
    \title{Homework 5 of Stochastic Processes}
    \author{姓名:林奇峰\qquad 学号:19110977}
    \maketitle

    \section{Exercise 4.2}
    Show that every Markov chain with $\text{M}<\infty$ states contains at least one recurrent set of states. Explaining each of the following statements is suffcient.
    \begin{enumerate}[(a)]
        \item if state $i_1$ is transient, then there is some other state $i_2$ such that $i_1\rightarrow i_2$ and $i_2\nrightarrow i_1$.
        \item if the $i_2$ of (a) is also transient, there is a third state $i_3$ such that $i_2\rightarrow i_3$, $i_3\nrightarrow i_2$; that state must satisfy $i_3\neq i_2,i_3\neq i_1$.
        \item Continue iteratively to repeat (b) for successive states $i_3$ such that $i_1, i_2,\dots.$ That is, if $i_1,\dots,i_k$ are generated as above and are all transient, generate $i_{k+1}$ such that $i_k\rightarrow i_{k+1}$ and $i_{k+1}\nrightarrow i_k$. Then $i_{k+1}\neq i_j$ for $1\leq j\leq k$.
        \item Show that for some $k\leq \text{M}$, $k$ is not transient, i.e., it is recurrent, so a recurrent exists. 
    \end{enumerate}

    \textbf{Solutions:}

    \begin{enumerate}[(a)]
        \item According to the \textbf{Definition 4.2.5}, if there is no such state $i_2$, $i_1$ is recurrent, which is a contradiction with the fact that $i_1$ is transient. Also, $i_2\neq i_1$ since otherwise $i_1\rightarrow i_2,i_2\rightarrow i_1$, which is recurrent.
        \item Firstly, $i_2\rightarrow i_3,i_3\nrightarrow i_2$ and $i_3\neq i_2$ can be proved by (a). 
        
        Secondly, $i_1\rightarrow i_2, i_2\rightarrow i_3$ implies that $i_1\rightarrow i_3$. if $i_3\rightarrow i_1$,  it implies that $i_3\rightarrow i_1, i_1\rightarrow i_2$ and $i_3\rightarrow i_2$, which is a contradiction with the fact $i_3\nrightarrow i_2$. Thus, $i_3\nrightarrow i_1$ and $i_3\neq i_1$ since otherwise $i_1\rightarrow i_3,i_3\rightarrow i_1$, which is recurrent.
        \item Firstly, the reason for which $i_{k+1}$ exists with $i_k\rightarrow i_{k+1}, i_{k+1}\nrightarrow i_k$ and $i_{k+1}\neq i_k$ can be proved by (a). 
        
        Secondly, for $1\leq j\leq k$, if $i_{k+1}=i_j$, it implies that $i_j\rightarrow i_{k+1}, i_{k+1}\rightarrow i_j, i_j\rightarrow i_k$ and $i_{k+1}\rightarrow i_k$, which is a contradiction with the fact $i_{k+1}\nrightarrow i_k$. Thus, $i_{k+1}\neq i_j$ for $1\leq j\leq k$.
        \item If all states are transient, it means that $k=\text{M}$ and there exists a state $i_{\text{M}+1}$ with $i_{\text{M}+1}\neq i_j$ for $1\leq j\leq \text{M}$, which is a contradiction with the fact there exists only $\text{M}$ states. Therefore, there mus be some $k\leq\text{M}$, $i_k$ is not transient. Thus, a recurrent class exists.
    \end{enumerate}

    \section{Exercise 4.3}
    Consider a finite-state Markov chain in which some given state, say state $1$, is accessible from every other state. Show that the chain has exactly one recurrent class $\mathcal{R}$ of states and state $1\in\mathcal{R}$. (Note that the chain is then a unichain.)

    \textbf{Solutions:}
    
    Firstly, there is no state $i$ such that $1\rightarrow i$ and $i\nrightarrow 1$ since $1$ is accessible from every other states. Therefore, the state $1$ is recurrent.
    
    Secondly, for any given state $i$, if $1\nrightarrow i$, then $i$ must be transient since $i\rightarrow 1$; if $1\rightarrow i$, then $1\leftrightarrow i$ and $i$ must be in the same recurrent class as 1.

    Thus, each state is either transient or in the same recurrent class as $1$.

    \section{Exercise 4.8}
    A transition probability matrix [P] is said to be doubly stochastic if 
    \begin{equation*}
        \sum_j P_{ij}=1\quad\text{for all }i ,\quad\sum_i P_{ij}=1\quad\text{for all j.}
    \end{equation*}
    \

    That is, the row sum and the column sum each equal 1. If a doubly stochastic chain has $\text{M}$ states and is ergodic (i.e., has a single class of states and aperiodic), calculate its steady-state probabilities.

    \textbf{Solutions:}
    
    TO-DO
\end{document}